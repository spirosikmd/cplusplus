\documentclass[a4paper]{article}
\usepackage{latexsym}
\usepackage[a4paper]{geometry}
\usepackage{color}
\usepackage{listings}
\usepackage[pdftex]{graphicx}
\usepackage{subfig}

\definecolor{Blue}{rgb}{0,0,0.5}
\definecolor{Green}{rgb}{0,0.75,0.0}
\definecolor{LightGray}{rgb}{0.6,0.6,0.6}
\definecolor{DarkGray}{rgb}{0.3,0.3,0.3}
%\lstset{language=C++,
%   keywords={function,uint8,uint16,uint32,double,break,case,catch,continue,else,elseif,end,for,global,if,otherwise,persistent,return,switch,try,while},
%   basicstyle=\ttfamily\small,
%   breaklines=true,
%   keywordstyle=\bfseries\color{Blue},
%   commentstyle=\itshape\color{LightGray},
%   stringstyle=\color{Green},
%   numbers=left,
%   numberstyle=\tiny\color{DarkGray},
%   stepnumber=1,
%   numbersep=10pt,
%   backgroundcolor=\color{white},
%   tabsize=2,
%   showspaces=false,
%   showstringspaces=false,
%   captionpos=b}

%Boldface text for type writer font
\usepackage{bold-extra} %\DeclareFontShape{OT1}{cmtt}{bx}{n}{<5><6><7><8><9><10><10.95><12><14.4><17.28><20.74><24.88>cmttb10}{}

%Break words properly at the end of a line (which isn't sloppy...)
\sloppy

%Use command \exercise for each exercise
\newcounter{exerciseCount}
\setcounter{exerciseCount}{0}
\newcommand{\exercise}[1]{\addtocounter{exerciseCount}{1} \noindent \medskip {\large \textsf{\textbf{Exercise \arabic{exerciseCount} #1}}} \par}
\renewcommand{\theenumi}{\textsf{\textbf{\alph{enumi}}}}

%Use command \code for code snippets
\newcommand{\code}[1]{\textnormal{\texttt{#1}}}

\setlength{\parindent}{0in}

\title{\textsf{C/C++ Course \\ Chapter 1: Set One: Introduction}}
\author{Christian Manteuffel \and Spyros Ioakeimidis}
\date{\today}

\begin{document}
\maketitle

\exercise{}

% not complete descriptions
\begin{description}
	\item[hello.cc]
	This is the C++ source file.
	\item[hello.o]\-\\
	This is the object file which is generated by the compiler. It contains external references, and definitions as instructions.
	\item[hello]\-\\
	This is the executable program, which is generated at the linking phase.
\end{description}

\lstinputlisting[language=C++,caption={hello.cc}]{./src/hello.cc}



Command for compiling the program: g++-4.7.1 -Wall --std=c++11 -c hello.cc\\

Command for linking the program: g++-4.7.1 -Wall --std=c++11 -o hello hello.o\\

Output: 
\begin{verbatim}
Hello World
\end{verbatim}

The size of the source file hello.cc is 69Bytes, the object file hello.o is 1824Bytes and the executable hello is 9300Byte large. After stripping the symbol tables the executable is 8988Bytes large. The file iostream has been found under \verb|/usr/gcc-4.7.1/include/c++/4.7.1| and is 2735Bytes large.\\

\exercise{}

We identified 13 basic data types, although \texttt{string} is a class-type.
\begin{verbatim}
void; char; short; int; long; float; double; bool; wchar_t; 
long long; long double; long long int; string
\end{verbatim}


\exercise{}
%spyros
\begin{description}
	\item[b / B] indicates a hexadecimal value, e.g.\ 0xb which is equal to decimal 11. It can also be used to define a binary constant, e.g. 0b1001 which is equal to the decimal value 9.
	\item[e / E]
	\item[F]
	\item[L]
	\item[p]
	\item[U]
	\item[x / a / f]
\end{description}

\exercise{}
%cm
The character a is represented by the decimal value 97 in the ASCII -Table. Three times 97 will be larger than 256 and thus exceeds the size of the char datatype, which results in an implicit overflow $(3 * 97) \% 256$.

After the overflow the char has the decimal value 35 or the character representation \#.  

\exercise{}
%spyros

\exercise{}
%cm
\exercise{}
%spyros
\exercise{}
%cm

\end{document}
