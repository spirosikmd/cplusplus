\documentclass[a4paper]{article}
\usepackage{latexsym}
\usepackage[a4paper]{geometry}
\usepackage{color}
\usepackage{listings}
\usepackage[pdftex]{graphicx}
\usepackage{subfig}

\definecolor{Blue}{rgb}{0,0,0.5}
\definecolor{Green}{rgb}{0,0.75,0.0}
\definecolor{LightGray}{rgb}{0.6,0.6,0.6}
\definecolor{DarkGray}{rgb}{0.3,0.3,0.3}
%\lstset{language=C++,
%   keywords={function,uint8,uint16,uint32,double,break,case,catch,continue,else,elseif,end,for,global,if,otherwise,persistent,return,switch,try,while},
%   basicstyle=\ttfamily\small,
%   breaklines=true,
%   keywordstyle=\bfseries\color{Blue},
%   commentstyle=\itshape\color{LightGray},
%   stringstyle=\color{Green},
%   numbers=left,
%   numberstyle=\tiny\color{DarkGray},
%   stepnumber=1,
%   numbersep=10pt,
%   backgroundcolor=\color{white},
%   tabsize=2,
%   showspaces=false,
%   showstringspaces=false,
%   captionpos=b}

%Boldface text for type writer font
\usepackage{bold-extra} %\DeclareFontShape{OT1}{cmtt}{bx}{n}{<5><6><7><8><9><10><10.95><12><14.4><17.28><20.74><24.88>cmttb10}{}

%Break words properly at the end of a line (which isn't sloppy...)
\sloppy

%Use command \exercise for each exercise
\newcounter{exerciseCount}
\setcounter{exerciseCount}{0}
\newcommand{\exercise}[1]{\addtocounter{exerciseCount}{1} \noindent \medskip {\large \textsf{\textbf{Exercise \arabic{exerciseCount} #1}}} \par}
\renewcommand{\theenumi}{\textsf{\textbf{\alph{enumi}}}}

%Use command \code for code snippets
\newcommand{\code}[1]{\textnormal{\texttt{#1}}}

\setlength{\parindent}{0in}

\title{\textsf{C/C++ Course \\ Chapter 1: Set One: Introduction}}
\author{Christian Manteuffel \and Spyros Ioakeimidis}
\date{\today}

\begin{document}
\maketitle

\exercise{}

% not complete descriptions
\begin{description}
	\item[hello.cc]
	This is the C++ source file.
	\item[hello.o]\-\\
	This is the object file which is generated by the compiler. It contains external references, and definitions as instructions.
	\item[hello]\-\\
	This is the executable program, which is generated at the linking phase.
\end{description}

\lstinputlisting[language=C++,caption={hello.cc}]{./src/hello.cc}



Command for compiling the program: g++-4.7.1 -Wall --std=c++11 -c hello.cc\\

Command for linking the program: g++-4.7.1 -Wall --std=c++11 -o hello hello.o\\

Output: 
\begin{verbatim}
Hello World
\end{verbatim}

The size of the source file hello.cc is 69Bytes, the object file hello.o is 1824Bytes and the executable hello is 9300Byte large. After stripping the symbol tables the executable is 8988Bytes large. The file iostream has been found under \verb|/usr/gcc-4.7.1/include/c++/4.7.1| and is 2735Bytes large.\\

\exercise{}

We identified 13 basic data types, although \texttt{string} is a class-type.
\begin{verbatim}
void; char; short; int; long; float; double; bool; wchar_t; 
long long; long double; long long int; string
\end{verbatim}


\exercise{}
%spyros
\begin{description}
	\item[b / B] indicates a hexadecimal value, e.g.\ 0xb which is equal to decimal 11. It can also be used to define a binary constant, e.g. 0b1001 which is equal to the decimal value 9.
	\item[e / E] is the exponentiation character in floating point literal values. For example, 1.15E+2 represents the decimal value 115. It can also be used to indicate a hexadecimal value, e.g. 0xE which is equal to the decimal value 14.
	\item[f / F] can be used as a postfix to a non-integral numeric constant to indicate a value of type float. For example, 10.F is a float value. Another example is 1.155E+2F which is equal to the decimal value 115.5. It also indicates a hexadecimal value, e.g. 0xf which is equal to the decimal value 15.
	\item[L] can be used as a prefix to indicate a character string whose elements are wchar\_t type characters. For example, L"Hello, World!". It can also be used as a postfix to an integral value to indicate a value of type long, e.g. 20L.
	\item[p] is used to specify the power in hexadecimal floating point numbers. For example, 0x10p2 which is equal to the decimal value 64. The exponent itself is read as a decimal constant and can therefore not start with 0x.
	\item[U] can be used as postfix to an integral value to indicate an unsigned value. For example, 2U. It can also be combined with L to produce an unsigned long int value, e.g. 10UL.
	\item[x / X] is used in order to specify hexadecimal constants, e.g. 0x8 which is equal to the decimal value 8.
	\item[a / A] is used in order to specify a hexadecimal value, e.g. 0xA which is equal to the decimal value 10.
	\item[c / C] is used in order to indicate a hexadecimal value, e.g. 0xC which is equal to the decimal value 12.
	\item[d / D] is used in order to indicate a hexadecimal value, e.g. 0xC which is equal to the decimal value 13.
\end{description}

\exercise{}
%cm
The character a is represented by the decimal value 97 in the ASCII -Table. Three times 97 will be larger than 256 and thus exceeds the size of the char datatype, which results in an implicit overflow $(3 * 97) \% 256$.

After the overflow the char has the decimal value 35 or the character representation \#.\\

\exercise{}
%spyros

The type of the parameters of the function \verb|showDate| are not well chosen and the design can be improved. As an alternative design we could have created an enumeration for the type of the month. The definition of the enumeration could be the following:

\begin{verbatim}
enum Month
{
    January=1, February, March, April, May, June, July,
    August, September, October, November, December
};
\end{verbatim}

Now the declaration of the function will be \verb|showDate(int daynr, Month month, int year)|, letting the user specify the name of the month instead of a decimal number, e.g. \verb|showDate(2, September, 2010)|. This leads to better readable and maintainable code.\\

There is no need to change the year, as it is perceived by definition an integer type. The same could be true for the number of the day of the month. However, in order to be able to generate the name of the day in the output, we assume that the function definition contains a way of accomplishing this.\\

\exercise{}
%cm
\exercise{}
%spyros

\begin{description}
	\item[Q1]\-
	The source file will always compile and without warnings. Even though in some cases the same integer numbers are used in order to define a company, this is not an issue for the compiler as long as the names of the companies remain unique.
	\item[Q2]\-
	% make the define lists enumerations
	\item[Q3]\-
	
\end{description}

\exercise{}
%cm

\end{document}
