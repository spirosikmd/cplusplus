\documentclass[a4paper]{article}
\usepackage{latexsym}
\usepackage[a4paper]{geometry}
\usepackage{color}
\usepackage{listings}
\usepackage[pdftex]{graphicx}
\usepackage{subfig}

\definecolor{Blue}{rgb}{0,0,0.5}
\definecolor{Green}{rgb}{0,0.75,0.0}
\definecolor{LightGray}{rgb}{0.6,0.6,0.6}
\definecolor{DarkGray}{rgb}{0.3,0.3,0.3}
\definecolor{gray}{rgb}{0.45,0.45,0.45}
\usepackage{caption}
\DeclareCaptionFont{white}{\color{white}}
\DeclareCaptionFormat{listing}{%
\parbox{\textwidth}{\colorbox{gray}{\parbox{\textwidth}{#1#2#3}}\vskip-4pt}}
\captionsetup[lstlisting]{format=listing,labelfont=white,textfont=white}

\lstset{language=C++,
%   keywords={function,uint8,uint16,uint32,double,break,case,catch,continue,else,elseif,end,for,global,if,otherwise,persistent,return,switch,try,while},
   basicstyle=\ttfamily\small,
%   breaklines=true,
   keywordstyle=\bfseries,
   commentstyle=\itshape\color{LightGray},
   stringstyle=\itshape\color{DarkGray},
%   numbers=left,
%   numberstyle=\tiny\color{DarkGray},
%   stepnumber=1,
%   numbersep=10pt,
   backgroundcolor=\color{white},
%   tabsize=2,
   showspaces=false,
   showstringspaces=false,
   captionpos=t,
   frame=lrb,
   xleftmargin=\fboxsep,
   xrightmargin=-\fboxsep,
   %belowcaptionskip=1em
   }

%Boldface text for type writer font
\usepackage{bold-extra} %\DeclareFontShape{OT1}{cmtt}{bx}{n}{<5><6><7><8><9><10><10.95><12><14.4><17.28><20.74><24.88>cmttb10}{}

%Break words properly at the end of a line (which isn't sloppy...)
\sloppy

%Use command \exercise for each exercise
\newcounter{exerciseCount}
\setcounter{exerciseCount}{8}
\newcommand{\exercise}[1]{\addtocounter{exerciseCount}{1} \noindent \medskip {\large \textsf{\textbf{Exercise \arabic{exerciseCount} #1}}} \par}
\renewcommand{\theenumi}{\textsf{\textbf{\alph{enumi}}}}

%Use command \code for code snippets
\newcommand{\code}[1]{\textnormal{\texttt{#1}}}

\setlength{\parindent}{0in}

\title{\textsf{C/C++ Part I \\ Chapter 2 - Set Two}}
\author{Christian Manteuffel \and Spyros Ioakeimidis}
\date{\today}

\begin{document}
\maketitle

\exercise{}
%spyros

\lstinputlisting[caption=number.cc]{./src/ex9/number.cc}

When we assign 100 to the variable \verb|value| the character `d' is shown. This because the decimal number 100 corresponds to the character `d' in the ascii table. The type \verb|size_t| is 8 bytes and the type \verb|char| is 1 byte. So when the variable \verb|value| with 356 is casted from \verb|size_t| to \verb|char|, it results to a variable with the size of 1 byte.

The byte that is left has the binary form of \verb|01100100| which is equal to 100 and hence it shows the same character `d', because as we previously mentioned the decimal value 100 corresponds to the character `d' in the ascii table.\\

The unsigned value of 200 is shown using the variable \verb|value|. In the first case we casted the variable \verb|value| to the type \verb|char|, because we knew that this conversion would give us directly the character. Although, we were aware that this conversion would lead to loss of data, this was not an issue in this case. The \verb|char| type has sufficient size in order to correspond to every character from the ascii table.

In the second case, we casted the variable \verb|ch| to \verb|size_t|. We used this cast because we knew that this would give us back the number that corresponds to the character `d'. We could have used a cast to \verb|unsigned int|, however we preferred to use \verb|size_t| instead, as it is also proposed in the C++ Annotations.\\

\exercise{}
%manni
% Purpose of this exercise: understand how the compiler reads source files and understand how the decrement operators work.
% Consider the following expressions (make sure you realize the correct number of minuses, hence the top line indicating column numbers):
% 
% 
%             123456
%     1:      ----a
%     2:      -----a
%     3:      a----
%         
% Why does the first expression compile?
% Why does the second expression not compile?
% Why does the third expression not compile?
% Change the a in the first expression into the value 5 and explain why the thus changed expression does not compile.
% Change the layout of the second expression showing the way the compiler interprets that expression: add blanks to show what different elements the compiler sees.
% Provide two different layouts for the second expression which would result in completely different (but compilable) interpretations of the expression.
% When changing the layout, do not use parentheses but only blanks.

\begin{description}
    \item[a)]\- 
    	\verb|----a| The compiler sees two decrement prefix operators. The prefix operator immediately changes the value of \verb|a| instead of returning a temporary value, hence it is ok to apply multiple prefix operators.
    \item[b)]\- 
    	\verb|-----a| no solution found.... 
    \item[c)]\- 
	    \verb|a----|  The prefix operators return an lvalue, which can be chained. The postfix operator return a prvalue, which cannot be used by operators that modify a value\footnote{cf. ISO/IEC 14882-2011 Section 5.2.6}. Furthermore, it is also logically questionable. Assumed that a is 5, the first decrement would compute the value 4 but return a temporary 5. Hence, the second decrement would be applied to the temporary 5\verb|--| again. 
    \item[d)]\- 
    	\verb|----5| results in the error \texttt{lvalue required as decrement operand}. The operator cannot be applied to the value 5 because the computed result cannot be stored. 
    \item[e) Change the layout of b the way the compiler interprets it]\-    	\verb|-- -- -a|

    \item[f) Two different compilable layouts for the b]\-

	\verb|- ----a| results in two decrements and turns the result negative, e.g. if a is initially 5, after the operation it will be -3.\\
    \verb|- - - --a| results in one decrement and turns the result negative, positive and negative again, e.g. if a is initially 5, after the operation it will be -4.
\end{description}

\exercise{}
%spyros

\lstinputlisting[caption=path.cc]{./src/ex11/path.cc}


\exercise{}

If \verb|getline| does not encounter a \textbackslash{n}, the eofbit will be raised and \verb|cin.eof()| returns true (cf. C++Annotations p. 85). 

\lstinputlisting[caption=line.cc,label=lst:line]{src/ex12/line.cc}
\begin{description}
\item[Would you use gets to fill that buffer?]\-

No we would not use the \verb|gets|-function because it has security issues.  The function does not check, if the buffer is large enough respectively it does not stop reading if the buffer is full. Hence, the function could result in a buffer overflow\footnote{http://www.cplusplus.com/reference/clibrary/cstdio/gets/}. In contrast, C++ strings handle all the required memory management, which helps in developing more robust applications. 


\end{description}
\exercise{}
%spyros

\begin{itemize}
	\item \verb|~ a ^ 012 & x << 4|
	\begin{verbatim}
	|--                 1
	            --|---  2
	      ----|-------  3
	----|-------------  4
	\end{verbatim}
	\item \verb|x += y = 4 == a|
	\begin{verbatim}
	         --|---  1
	--|---           2
	-------|-------  3
	\end{verbatim}
	\item \verb|a == b == c|
	\begin{verbatim}
	--|---       1
	-------|---  2
	\end{verbatim}
	
	However, the above expression is wrong and it should not be used like this.
\end{itemize}

\exercise{}
% manni (optional)
A number is an exact power of two, if and only if the binary representation contains exactly one 1 since each digit represents an increasing power of 2, e.g. 4 = 0x0100. This can be used to check for the power of two using the binary and operator (\verb|&|).
The result of the (\verb|&|)-operator between the value and the value-1 will only result in 0, if value is a power of two. In all other cases the result will contain at least two 1's (or zero).


The code shown in Listing~\ref{lst:powerOfTwo} first checks if the value is not 0 and if this is not the case, uses the perviously discussed operation to determine whether a number is an exact power of two. 

\lstinputlisting[caption=powerOfTwo.cc,label=lst:powerOfTwo]{src/ex14/powerOfTwo.cc}

\exercise{}
%spyros

\lstinputlisting[caption=coordinate.cc]{./src/ex15/coordinate.cc}

\end{document}
