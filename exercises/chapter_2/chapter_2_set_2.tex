\documentclass[a4paper]{article}
\usepackage{latexsym}
\usepackage[a4paper]{geometry}
\usepackage{color}
\usepackage{listings}
\usepackage[pdftex]{graphicx}
\usepackage{subfig}

\definecolor{Blue}{rgb}{0,0,0.5}
\definecolor{Green}{rgb}{0,0.75,0.0}
\definecolor{LightGray}{rgb}{0.6,0.6,0.6}
\definecolor{DarkGray}{rgb}{0.3,0.3,0.3}
\lstset{language=C++,
%   keywords={function,uint8,uint16,uint32,double,break,case,catch,continue,else,elseif,end,for,global,if,otherwise,persistent,return,switch,try,while},
%   basicstyle=\ttfamily\small,
%   breaklines=true,
%   keywordstyle=\bfseries\color{Blue},
%   commentstyle=\itshape\color{LightGray},
%   stringstyle=\color{Green},
%   numbers=left,
%   numberstyle=\tiny\color{DarkGray},
%   stepnumber=1,
%   numbersep=10pt,
%   backgroundcolor=\color{white},
%   tabsize=2,
   showspaces=false,
   showstringspaces=false,
   captionpos=t}

%Boldface text for type writer font
\usepackage{bold-extra} %\DeclareFontShape{OT1}{cmtt}{bx}{n}{<5><6><7><8><9><10><10.95><12><14.4><17.28><20.74><24.88>cmttb10}{}

%Break words properly at the end of a line (which isn't sloppy...)
\sloppy

%Use command \exercise for each exercise
\newcounter{exerciseCount}
\setcounter{exerciseCount}{8}
\newcommand{\exercise}[1]{\addtocounter{exerciseCount}{1} \noindent \medskip {\large \textsf{\textbf{Exercise \arabic{exerciseCount} #1}}} \par}
\renewcommand{\theenumi}{\textsf{\textbf{\alph{enumi}}}}

%Use command \code for code snippets
\newcommand{\code}[1]{\textnormal{\texttt{#1}}}

\setlength{\parindent}{0in}

\title{\textsf{C/C++ Part I \\ Chapter 2 - Set Two}}
\author{Christian Manteuffel \and Spyros Ioakeimidis}
\date{\today}

\begin{document}
\maketitle

\exercise{}
%spyros

\lstinputlisting{./src/ex9/ex9.cc}

The value of 100 shows the character `d'  because the decimal number 100 corresponds to the character `d' in the ascii table. The value of 356 shows also the character `d' because ...

size\_t is 8 bytes and char is 1 byte, so when 356 is casted from size\_t to char only 1 byte is left ...

the byte that is left has the binary form of 01100100 which is equal to 100 and hence it shows the same character `d' because as previously mentioned the decimal value 100 corresponds to the character `d' in the ascii table\\

\exercise{}
%manni

\exercise{}
%spyros

\exercise{}
%manni

\exercise{}
%spyros

\exercise{}
%manni

\exercise{}
%spyros

\end{document}
